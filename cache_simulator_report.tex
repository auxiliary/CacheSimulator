\documentclass[11pt]{article}
%Gummi|063|=)
\title{\textbf{Cache Simulator Report}}
\author{Alok Hota, Mohammad Ahmadzadeh\\\href{https://github.com/auxiliary/CacheSimulator}{https://github.com/auxiliary/CacheSimulator}}

\usepackage{amsthm}
\usepackage{hyperref}
\usepackage{amsmath}
\usepackage{tikz}
\usepackage{graphicx}
\usepackage{placeins}
\setlength{\parindent}{0pt}
\usepackage[margin=0.5in]{geometry}
\begin{document}

\maketitle

\section{Functionalities}
\paragraph{}
Our cache simulator takes a configuration file and a trace file from the command line. It then generates a memory hierarchy based on the configurations and runs the instructions read from the trace file. 

\subsection{Features}
\paragraph{}
The following features have been implemented in the cache simulator:
\begin{itemize}
	\item Fully configurable cache hierarchy using YAML
		\begin{itemize}
			\item Architecture details, level 1 cache and main memory are required
			\item Level 2 and level 3 are optional
		\end{itemize}
	\item Supports write through and write back
	\item Optionally draws cache layout after simulation
	\item Writes simulation results to a log file and standard output
\end{itemize}

\subsection{Assumptions}
\begin{itemize}
	\item When a new word is written to level 1 cache, a new block is allocated. If the CPU then reads another word from that block it will be reading an empty word. The simulator will count this as a read hit. This would actually result in a segfault but we consider this out of scope. 
\end{itemize}

\section{Testing}
\paragraph{}
We tested the simulator with several different cache configurations and trace files as well as a stress test. The stress test is generated with a python script located in gen\_test/gen\_stress\_test.py and provides a large randomly generated trace file. 

\section{Conclusion and results}
\paragraph{}

\end{document}
